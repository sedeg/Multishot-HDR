\documentclass[liststotoc,bibtotoc,fontsize=14pt,]{scrreprt}
\usepackage[utf8]{inputenc} % Zeichenkodierung
\usepackage[ngerman]{babel} % neue deutsche Rechtschreibung
\usepackage{etoolbox}
\setlength{\footskip}{30pt}
\apptocmd{\thebibliography}{\raggedright}{}{}
\usepackage{graphicx}
\usepackage{caption}
\usepackage{subcaption}
\usepackage{url}
\usepackage[onehalfspacing]{setspace}
\usepackage{breakurl}
\usepackage{float}
\usepackage[table,xcdraw]{xcolor}
\usepackage{tabularx}
\usepackage[breaklinks]{hyperref}
\def\UrlBreaks{\do\/\do-}
\usepackage{tocloft}
\usepackage{chngcntr}
\usepackage{listings}
\usepackage{color}
\usepackage[parfill]{parskip}
\definecolor{lightgray}{rgb}{.9,.9,.9}
\definecolor{darkgray}{rgb}{.4,.4,.4}
\definecolor{purple}{rgb}{0.65, 0.12, 0.82}

\counterwithout{footnote}{chapter}

\deffootnote[2em]{2em}{2em}{%
	\makebox[2em][l]{\bfseries\thefootnotemark}}

\renewcommand{\cftchapdotsep}{\cftdotsep}
\renewcommand{\cftchapleader}{\cftdotfill{\cftchapdotsep}}
\usepackage{amsmath}
\usepackage[paper=a4paper,left=30mm,right=30mm,top=25mm,bottom=25mm]{geometry}
\usepackage[section]{placeins}
\usepackage[font=small,justification=justified]{caption}
\newcommand{\namesigdate}[3][Ort, Datum]{%
	\parbox{\textwidth}{
		\raggedleft #3 
		\vspace{2cm}
		
		\parbox{5cm}{
			\raggedright
			\rule{6cm}{1pt}\\
			#1 
		}
		\hfill
		\parbox{5cm}{
			\raggedright
			\rule{6cm}{1pt}\\
			#2
		}
	}
}


\newcommand*{\tabularwidth}{}
\newdimen\tabularwidth
\usepackage{minitoc}
\hypersetup{
	colorlinks,
	citecolor=black,
	filecolor=black,
	linkcolor=black,
	urlcolor=black
}


\title{Dokumentation Panoramafotografie}
\author{Sebastian Degner}

\begin{document}
	%\maketitle
	
	\begin{titlepage}
		\begin{center}
			\vspace{2cm}
			Dokumentation\\ \textbf{Multishot-Technik in der digitalen Fotografie}\\ 
			\vspace{2,5cm}
			\includegraphics[width=5cm]{HTWK_Logo_RGB-transparent_250.png}\\
			
			\vspace{2,5cm}
			\huge \textbf{\textsf{Dokumentation HDR-Fotografie}} \\
			\vspace{3cm}
			\fontsize{15}{18} \textbf{Hochschule für Technik, Wirtschaft und Kultur
				Leipzig\\ Fakultät Informatik, Mathematik und Naturwissenschaften\\   Masterstudiengang Medieninformatik}\\
			\vspace{3cm}
		\end{center}
		\normalsize{
			\begin{tabular}{ll}
				Eingereicht von: & {Sebastian Degner} \\
				 & {Sebastian Knabe} \\
				Studiengang: & 15 MIM\\
				Eingereicht am: & 31. Januar 2017 \\
			\end{tabular}\\
		}
		
	\end{titlepage}
	
	\tableofcontents
	\clearpage
	\listoffigures
	\addcontentsline{toc}{chapter}{Abbildungsverzeichnis}

	\chapter{Einleitung}
	\label{ch:einleitung}

		
	\chapter{Aufnahmen}
	\label{ch:aufnahmen}
	
	\section{Leipziger Baumwollspinnerei}
	\label{sec:spinnerei}

	\subsubsection{Aufnahmeort und -idee}
			
	
		\subsubsection{Kameraeinstellungen}
		
		
	\subsubsection{Vorgehen und Fehleranalyse}
	
	\newpage
	\begin{figure}[h]
		\includegraphics[width=\linewidth]{img/ph.jpg}
		\caption{HDR-Aufnahme Leipziger Baumwollspinnerei}
	\end{figure}

	
	\section{City-Tunnel -- Wilhelm-Leuschner-Platz }
	\label{sec:tunnel}
	\subsubsection{Aufnahmeort und -idee}
	

	\subsubsection{Kameraeinstellungen}
		

	
	\subsubsection{Vorgehen und Fehleranalyse}


			 \newpage
			 \begin{figure}[h]
			 	\includegraphics[width=\linewidth]{img/ph.jpg}
			 	\caption{HDR-Aufnahme City-Tunnel -- Wilhelm-Leuschner-Pl. }
			 \end{figure}


	\section{Deutsche Nationalbibliothek}
	\label{sec:bibo}
	\subsubsection{Aufnahmeort und -idee}
	

	
	\subsubsection{Kameraeinstellungen}

			
	\subsubsection{Vorgehen und Fehleranalyse}
	
	
			 \newpage
			 \begin{figure}[h]
			 	\includegraphics[width=\linewidth]{img/ph.jpg}
			 	\caption{HDR-Aufnahme Deutsche Nationalbibliothek}
			 \end{figure}

	\section{Palmengarten Leipzig}
	\label{sec:palme}
	\subsubsection{Aufnahmeort und -idee}
	
		\subsubsection{Kameraeinstellungen}
	
		
	\subsubsection{Vorgehen und Fehleranalyse}


			 \newpage
			 \begin{figure}[h]
			 	\includegraphics[width=\linewidth]{img/ph.jpg}
			 	\caption{HDR-Aufnahme Palmengarten Leipzig}
			 \end{figure}

	
	\section{Nikolaikirche}
	\label{sec:nikolai}
		\subsubsection{Aufnahmeort und -idee}

		
		\subsubsection{Kameraeinstellungen}
			
		\subsubsection{Vorgehen und Fehleranalyse}
		
			 \newpage
			 \begin{figure}[h]
			 	\includegraphics[width=\linewidth]{img/ph.jpg}
			 	\caption{HDR-Aufnahme Nikolaikirche}
			 \end{figure}

	\section{Speck's Hof}
	\label{sec:speck}
			\subsubsection{Aufnahmeort und -idee}
					
			\subsubsection{Kameraeinstellungen}
		
			
			\subsubsection{Vorgehen und Fehleranalyse}
		
					 \newpage
					 \begin{figure}[h]
					 	\includegraphics[width=\linewidth]{img/ph.jpg}
					 	\caption{HDR-Aufnahme Speck's Hof}
					 \end{figure}
			

	\chapter{Vorbereitung}
	
	\section{Einzelbildaufnahmen}
	\label{sec:einzel}
	
		\section{Tone-Mapping}
		\label{sec:einzel}

	\chapter{HDR-Processing}
	\label{ch:processing}

	
	\section{Adobe Photoshop}
	\label{sec:photoshop}
	
	
	
	\section{Zweites Prog}
	\label{sec:prog}

	\chapter{Nachbearbeitung}
	\label{ch:nach}
	
	
\end{document}